\documentclass{exam}
\usepackage[utf8]{inputenc}

\begin{document}
%\printanswers
\begin{center}
\large Distributed Algorithms:\\ Leader election in a Asynchronous arbitrary network.
\end{center}
\vspace{5mm}
\begin{tabular}{l r}
    \makebox[0.7\textwidth]{Nome:\enspace\hrulefill} &
    \makebox[0.3\textwidth]{Matricola:\enspace\hrulefill} \\
\end{tabular}
\vspace{5mm}
 
\begin{questions}

\question{Can the following synchronous algorithms be ported to an asynchronous setting?}
\begin{parts}
\part{\emph{FloodMax}}
\begin{oneparchoices}
    \CorrectChoice{Yes}
    \choice{No}
\end{oneparchoices}
\part{\emph{OptFloodMax}}
\begin{oneparchoices}
    \choice{Yes}
    \CorrectChoice{No}
\end{oneparchoices}
\end{parts}

\question{What are the complexity results of the STtoLeader algorithm?}
\begin{parts}
\part{Message complexity: \fillin[n]}
\part{Time complexity: $\mathcal{O}(\fillin[n*(l+d)])$}
\end{parts}
%\begin{solutionordottedlines}[1in]
%    Solution. \\ Possibly long etc
%\end{solutionordottedlines}

\question{Which of the following are true about the AsynchSpanningTree? \emph{(there might be more than one.)}}
\begin{checkboxes}
    \choice{The height of the tree is always equal to the diameter.}
    \CorrectChoice{The message complexity is $\mathcal{O}(|E|)$}
    \CorrectChoice{It can be augmented with child pointers.}
    \choice{It doesn't need to have any process marked.}
\end{checkboxes}


\question{Which of these asynchronous algorithms terminate? \emph{(there might be more than one.)}}
\begin{checkboxes}
    \CorrectChoice{FloodMax}
    \choice{OptFloodMax}
    \CorrectChoice{AsynchSpanningTree}
    \choice{AsynchBFS}
    \choice{Bellman-Ford}
\end{checkboxes}

\question{There is no way to compute Breadth-First Search in an arbitrary network while also taking the shortest paths between nodes.}
\begin{choices}
    \choice{True}
    \CorrectChoice{False}
\end{choices}

\question{The Bellman-Ford algorithm always has at most polynomial time complexity.}
\begin{choices}
    \choice{True}
    \CorrectChoice{False}
\end{choices}

\end{questions}

\end{document}
