\documentclass{article}
\usepackage[utf8]{inputenc}
\begin{document}

    Assertion 1.
    \begin{description}
    \item[$receive$]{\small The receive of a search message to a node not in the spanning tree, allows for the node to enter the spanning tree by setting the parent variable to the neighbor that sent the message. It then sets the variables in order to send the search message to the remaining neighbors.\\
        If the node already has the parent variable set, it discards the message. 
    Since this only adds a node, $i_0$ is still contained in the spanning tree.}
    \item[$send$]{Since the send transition sends a message from the node i to the neighbors, we need
        to prove that i is in the spanning tree. By the algorithm, the send transition has the precondition of the $send(j)$ variable,
    which is set only as the node enters the spanning tree.}
    \item[$parent$]{The parent action will activate once, as the parent variable is set, reporting the node's presence in the spanning tree.}
    \end{description}

    Assertion 2.


    \begin{description}
    \item[$send$]{As the effect will remove the $search$ message from the $send(j)$ variable, the channel $C_{i,j}$ will now contain a $search$ message instead.}
    \item[$receive$]{As $parent_i$ becomes different than $null$, for each $j \in nbrs - {parent_i}$,
    it will set $send(j) = search$. For $parent_i$ instead, we can already prove from the first assertion that since a message came from $parent_i$ it must be in the spanning tree, and so its $parent$ variable different than $null$.}
    \item[$parent$]{It doesn't affect any of the variables in the assertion.}
    \end{description}


\end{document}
